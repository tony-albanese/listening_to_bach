\chapter{Chorales and Counterpoint}

\section{Chorales}
There is simply no getting around the fact that Bach was a devout Lutheran and that his unshakable faith in God permeated every sheet of his music. As a result, a large number of his works are religious in nature and were meant to serve as part of religious services. Having said that, I want you to understand that regardless of your religious views, there is something for everyone in these works. They can be enjoyed on a purely musical level. 

There is one particular piece of religious music called a chorale. A chorale is Lutheran hymn that is set to music. These originated with Martin Luther when he began to translate religious songs into his native German. Chorales melodies tend to be simple, direct and harmonized - accompanied by - the typical SATB. Let us listen to one of my favorites, "Was Gott tut dass ist wohlgetan". We'll here the tune first in its bare bones form and then hear it as Bach harmonized it.

%example - Was Gott tut ist wohlgetan.

Do you hear how Bach's harmonization gives shape to the melody and makes it so rich? There is more, though. Bach wrote many such harmonizations and we will indeed spend quite a bit of time talking about harmony. 

Bach, more than any other composer, both perfected and elevated the art of counterpoint. %Meaning of the word.
Counterpoint is art of combining two or more independent melodies so that they work together when played simultaneously. The idea behind good counterpoint is that all the lines of music are equally interesting and could, in theory, stand on their own. One does not necessarily have a clear melody and accompaniment. This is NOT how most music is today. Most music we are accustomed to has a melody on top and an accompaniment. %Example - Happy Birthday, any pop song.

The result is that when many listeners hear Bach's music, the ear, being used to monophonic music, tends to blur the musical lines together. The effect of the polyphony is then lost. However, with practice, you can train your ear to hear each line of music both separately and together.

In the following example, Bach uses the same chorale melody as part of a cantata. In this recording, the cello plays the chorale melody above a joyful orchestral melody. Just focus on trying to perceive both of these at the same time and savor how well they go together.

%Example: Was Gott tut ... Yo-yo ma recording Simply Baroque II

\section{Fugue}
Counterpoint is a general term to describe music consisting of multiple independent layers. A fugue, is a particular kind of counterpoint. A fugue is a composition in which one theme, called the subject, is developed and examined by a number of different lines called voices. In a fugue, the theme forms the center of the composition and the voices take turns examining, analyzing, and exploring the theme for all it's worth. Fugues are an intellectual musical structure - much like engineers working out exactly what a particular design can or cannot do.  We will not go into the structure of a fugue here, rather it is enough if you are able to spot one when you hear one. 


%Example of a fugue.

Bach was the undisputed master of fugue writing and his music is FULL of fugues. Other composers have written fugues, but they tend to sound dry and academic. Bach's fugues are MORE rigorous than any of his contemporaries or ancestors, but are full of beauty, wit, humor, and life. Let's listen to a few examples.

%A flat major from WTC
%Ehre sei dir Gott

%Fugues in concertos.

%Note on Art of Fugue and A Musical Offering