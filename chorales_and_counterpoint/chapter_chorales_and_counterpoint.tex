\chapter{Chorales and Counterpoint}
There is simply no getting around the fact that Bach was a devout Lutheran and that his unshakable faith in God permeated every sheet of music. As a result, a large number of his works are religious and were meant to serve the Church's purposes. Within many of his works is a chorale.

%Give definition and example - Was Gott tut ist wohlgetan.

Bach wrote many such harmonoizations and we will indeed spend quite a bit of time talking about harmony. For now, I simply wish you to appreciate how Bach puts flesh on the bare melody and gives it life.

But there is more, much more. Bach, more than any other composer, perfected and elevated the art of counterpoint. %Meaning of the word.
Counterpoint is art of combining two or more independent melodies so that they work together when played simultaneously. The idea behind good counterpoint is that all the lines of music are equally intersting. You don't necessarily have a clear meoldy and accompaniemnent. This is NOT how most music is today. Mot music we are accustomed to has a melody on top and an accompaniment. %Example
So when you hear Bach, your ear will smoosh everything together because that is what it has been doing. However, with practice, you can train your ear to hear each line of music both separately and together. You will hear much more. 