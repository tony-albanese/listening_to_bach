\chapter{Rhythm, Meter, and Bass}

\section{Meter and Rhythm}
Pitches themselves are not enough for your ear (and brain), to classify a group of sounds as music. The sounds must somehow be organized. The pitches, what you might call the notes, organize the sounds in terms of frequency - how high or low they are. Different pitches are in and of themselves not enough to make music. Rhythm and meter determine how this pitches, are grouped together. Meter, which we will not discuss too deeply here, refers to how the strength of the pulses in the music are grouped. Usually, this is in groups of 2, 3, or 4. %Give an exampe of counting and music in two and three.

 "Rhythm is music's pattern in time. Whatever other elements a given piece of music may have (e.g., patterns in pitch or timbre), rhythm is the one indispensable element of all music." In other words, rhythm refers to the pattern of the pitch duration. Think of Beethoven's fifth. Da-da-da-duuuuuuuh. Classic.
 
 Bach is the master of using rhythm and meter to make his music swing. It is hard to describe in words, so let us look at an example from the Magnificat. In the penultimate movement, Bach treats us to a short chorale fugue with the following text, 
 \begin{verse}
 Sicut locutus est ad patres nostros. Abraham et semini eus in secula.
 %Translate	
 \end{verse}

The pulse - the meter - is in groups of 2 throughout. It will provide an anchor. However, the rhythm is so interesting. Bach adds a syncopation on the word Abraham which adds a little variety. In addition,the pattern of sixteenth, then eight, than half notes, make of this figure allow the music to stretch and contract. It gives it a kind of motion and life.

\section{Bass}
The bass is the lowest section in the music in terms of pitch. The purpose of the bass is to support the music harmonically and rhythmically - more on that a bit later. For now, it is enough to understand that in most music, the bass lines are functional but rather boring. That does not mean the music is bad!
%Example - Brindisi Traviata
%Example - Vivaldi

One must keep in mind that Bach is the master of counterpoint. As a result, the basslines in a Bach composition are interesting and can stand alone musically. In the following examples, I want your ear to focus on how the bass lines have an interesting melody and rhythm. 

%Example third brandenburg towards the end
%Concerto for two violins.
%Ehre sei dir Gott Gesungen (Dir sei lob)

%Note - see if you can use a midi score to extract the bass line.