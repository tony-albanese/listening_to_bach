\chapter{Harmony and Key}
This is the dreaded music theory part of the talk. Please do not let the this topic discourage you. You do not need to be a musicologist (I'm not) to appreciate harmony and key. The labels that musicologists use to analyze a piece of music or to classify chords are good for their purposes. For us, our ears are sufficiently intelligent to perceive key and harmony in such a way as to make the music thoroughly excited. With a little guidance you can enhance your ears' innate ability to perceive key and harmony.

Key and harmony go hand-in-hand but they are not the same.

\section{Harmony}
Harmony is simply the combination of simultaneously sounded musical pitches. Terrific. A skilled composer knows how to combine these pitches in such a way that will give the music structure, meaning, and motion. Let is dive into what that means - what sound really is.

\subsection{Frequencies and Stability}
%Need a good image for the ear.
Let us begin with the instrument itself. When a musician draws a bow across a string, strikes a piano key, or blows on a horn, the instrument begins to vibrate. Those vibrations cause the air to vibrate.  Eventually that vibrating air pushes against the eardrum, which causes the three bones to vibrate, which causes the fluid inside the cochlea to vibrate, which causes the hair called cilia to vibrate, which sends an electrical signal to the brain along the auditory nerve. Sound is what we perceive when our brains receive electrical signal and interpret it. The key properties for that vibration are frequency (how often the air moves back and forth) and the amplitude (how hard). The latter determines how loud we perceive the pitch, the former, which we will focus on, determines how high or low it is. In other words, the frequency determines whether we hear an A, a G, a C#, or whatever.

%Give an example in Hz of what some pitches.

Great. Here is the point. Pitches which are in simple ratios of each other tend to us to sound pleasant or stable. Pitches whose frequencies are in more complex ratios will tend to sound unstable - dissonant.  %Give an example of octave, fifth, fourth
%Give example of minor third, tritone
When we hear a dissonant harmony, our ears will expect them to "resolve" into a combination that is more simple - more consonant. 
%Example - of resolution
What that means is a good composer will use dissonant harmonies to both create tension and release (like plot in literature) and for dramatic effect.

%As an example, use the opening of Brandenburg 3 to show how the harmonies, on the strong beats, move the music forward. Again, use the score.

%As an example of dramatic effect, There is the "Est ist vollbracht", the Magificicat, Omnes genrationes
%Concerto E-major violin/

\subsection{Key}
The concept of key is probably the most difficult for the non-musician to understand. It is not so much important for us to be able to identify which "key" a piece is in upon hearing. To do so requires lots of ear training. Rather, it is more important for us to understand what key is and how to detect key changes in the music. 

Let us dive into the concept of key. A collection of pitches does not determine a melody or any particular piece of music anymore than a collection of colors determines an image. Rather, it is the relationship from one pitch to the next that determines whether what we perceive is either noise or music. To see what I mean, look at the following images.

%Show examples of the same image with different color schemes, including black and white and sepia

Your eyes and brain clearly see that these images are the same even though the colors that determine these images are different. That is because the relationship from one pixel to the next is preserved. In the musical world, our brains can recognize a tune even if it is different keys. However, just as these images are the same, they might have different effects on us because of their color scheme. We might perceive one as being warmer, the other sharper, etc. It is no different with key.

When a composer is choosing a particular key for a piece, they are deciding two things. The first, they are saying this is the collection of pitches I will use in my composition. The second, they are going to decide how these pitches are related to each other. %Use c-major scale as an example.

For example, when a composer chooses C-major, they are saying, "I will use these collection of pitches in my composition." Further, they are implying that these pitches have roles to play. The most important is the first, called the tonic, and the fifth, the dominant. You see, each pitch is related to the others by its function in the key. The tonic is what our brains will perceive as the home, or the rest. The dominant pitch will be sensed as a departure and will naturally pull our ears towards the dominant. Let me show you what I mean. 
%Demonstrate tonic dominant relationship.

By themselves, the pitches have no meaning, but in relationship to each other, they can give the sense of motion, of logical conclusion. A good composer knows how to exploit these relationships to make the music go forward, to give it a sense of logic and progression, and of course surprise when our expectations are violated.

Of course, a composer will not want to keep the music in the same key for the whole piece. To do so will give the music a sense of stagnation. For example, 
%Give example of music that stays in one key.

However, a composer will modulate - to shift the music to a different key to create a sense of tension, drama, action, and then maybe resolution.
%Examples of modulation, Bach e-major concerto, brandenburg 3, beeethoven 2 piano concerto

What is important for the listener is not to be able to say, "Ah, that is G-major, or that is in C# minor." Rather, it is to perceive that the music is shifting and to enjoy the sense of action it creates.

\subsection{Relative Keys}
There is one other thing that you should be aware of at this stage and that is that keys are also related to each other via major and minor scales. Major and minor refer to the sizes of the intervals between each of the pitches in a scale. Major keys tend to sound brighter and minor keys tend to sound darker. Every major key has a relative minor key and vice versa. That means when the relationships between the pitches in a major scale are reorganized, you have a minor scale. Let me show you:

%C major and Aminor
%B flat major and g-minor
%Mention not 100 percent accurate, but good enough for now.

A composer will often go to the relative major or minor to show the theme in a different light.
%Example - Gminor keyboard concerto - movement 3 - going from g minor to b flat major
