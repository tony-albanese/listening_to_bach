\chapter{Harmony and Key}
This is the dreaded music theory part of the talk. Please do not let the this topic discourage you. You do not need to be a musicologist (I'm not) to appreciate harmony and key. The labels that musicologists use to analyze a piece of music or to classify chords are good for their purposes. For us, our ears are sufficiently intelligent to perceive key and harmony in such a way as to make the music thoroughly excited. With a little guidance you can enhance your ears' innate ability to perceive key and harmony.

Key and harmony go hand-in-hand but they are not the same.

\section{Harmony}
Harmony is simply the combination of simultaneously sounded musical pitches. Terrific. A skilled composer knows how to combine these pitches in such a way that will give the music structure, meaning, and motion. Let is dive into what that means - what sound really is.

\subsection{Frequencies and Stability}
%Need a good image for the ear.
Let us begin with the instrument itself. When a musician draws a bow across a string, strikes a piano key, or blows on a horn, the instrument begins to vibrate. Those vibrations cause the air to vibrate.  Eventually that vibrating air pushes against the eardrum, which causes the three bones to vibrate, which causes the fluid inside the cochlea to vibrate, which causes the hair called cilia to vibrate, which sends an electrical signal to the brain along the auditory nerve. Sound is what we perceive when our brains receive electrical signal and interpret it. The key properties for that vibration are frequency (how often the air moves back and forth) and the amplitude (how hard). The latter determines how loud we perceive the pitch, the former, which we will focus on, determines how high or low it is. In other words, the frequency determines whether we hear an A, a G, a C#, or whatever.

%Give an example in Hz of what some pitches.

Great. Here is the point. Pitches which are in simple ratios of each other tend to us to sound pleasant or stable. Pitches whose frequencies are in more complex ratios will tend to sound unstable - dissonant.  %Give an example of octave, fifth, fourth
%Give example of minor third, tritone
When we hear a dissonant harmony, our ears will expect them to "resolve" into a combination that is more simple - more consonant. 
%Example - of resolution
What that means is a good composer will use dissonant harmonies to both create tension and release (like plot in literature) and for dramatic effect.

%As an example, use the opening of Brandenburg 3 to show how the harmonies, on the strong beats, move the music forward. Again, use the score.

%As an example of dramatic effect, There is the "Est ist vollbracht", the Magificicat, Omnes genrationes
%Concerto E-major violin/

\subsection{Key}

