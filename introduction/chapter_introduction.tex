
\chapter{Introduction}

Thank you for being here this evening. I am fully aware that each and every one of you could be anywhere else this evening, and you have chosen to spend it with me. I take such a responsibility seriously and therefore hope that your trust in me will indeed pay you dividends.

\section{Framework}
That being said, I would like to frame this evening's lecture so that you can see the path our journey into Bach's music will take. The first thing I urge you to do is to trust me in managing the time. Please take a moment to silence the devices, to put whatever concerns you have to the side for a moment, and transfer the responsibility of time to me.

\section{The Intelligent Ear}
Our world is focused on what the eye can see. The great Israeli conductor Daniel Barenboim spoke about the ear as being an intelligent organ. In fact, he pointed out that in the uterus, the ear begins to process sounds long before the eye begins to process light. However, it what we see that is given precedence. The ear, though, is an intelligent organ and can process and interpret sound in the most fascinating way. We all possess the equipment to understand music on quite a deep level. I want to show that our ears, like other body parts, not only serve a practical function, but are also fun to use!

\section{Action in Music}
Since our world is dominated by what we see, I am not surprised when people tell me that classical music is boring. Especially in the music of Bach. Sure, there are pretty tunes and the like, but most of us are not really skilled in the art of listening. Therefore, at a concert, the eye is bored because it is desperately seeking action when all of the action is in the music. I will show you that Bach's music is indeed full of action - if you know what to listen for.